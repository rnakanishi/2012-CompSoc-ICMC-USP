\documentclass[a4paper,10pt]{article}
\usepackage[utf8]{inputenc}
\usepackage[brazil]{babel}

%opening
\title{Como o spam afeta a comodidade do correio eletrônico}
\author{Bianca Madoka Shimizu Oe 6810554\\Gustavo Shinji Inoue   6878758\\Rafael Umino Nakanishi 6793198}

\begin{document}

\maketitle
\newpage

\section{}
\subsection{Estudo de caso}

    Juliano é professor de uma universidade renomada.
    
    Em determinado semestre, ele ficou responsável por
ministrar a disciplina X e delegou a tarefa de auxiliá-lo
(como monitor bolsista) a Eduardo. Eduardo, por sua vez,
é pós-graduando sob a tutoria de Juliano e está na iminência
de defender sua tese de mestrado.

    A média final da disciplina é composta por $n$ projetos
(para serem feitos em grupos) de complexidade média e por
nenhuma prova. Os projetos são lançados ao longo do semestre
e cada um deles fica responsável por cobrir parte do conteúdo
programático.

    Os alunos passaram por uma provação para estarem onde estão
agora e desejam passar na matéria. Para tanto, é necessário
não apenas codificar o que lhes foi solicitado, mas também
elaborar um relatório altamente detalhado, que descreve os
resultados pedanticamente e comporá boa parte da nota. Esse
relatório é composto por um conjunto de regras pré-selecionadas
a dedo pelo professor Juliano e será corrigido pelo monitor
responsável.

    É aí que começam os a aparecer os problemas. Os projetos,
que eram para serem de complexidade média, acabam tendo sua
complexidade exponenciada por terem sido idealizados pelo monitor
(que não experiência com docência e os elabora independentemente
de considerar quanto tempo seria despendido para sua execução.)

    Cada problema identificado deve ser relatado via email ao
professor com cópia para o monitor para que seja sanado o mais
brevemente possível. No entanto, leva-se de 2 a 3 dias para que
qualquer um deles dê um retorno ao aluno.

    Como o professor acredita que cada dúvida levantada pode ser
compartilhada pela sala, cada solução é enviada para todos (também
por email).

    A partir do prazo de entrega, é definido pelo professor um horário
para que cada grupo apresente os resultados obtidos na forma de uma
apresentação. Aí reside um outro problema: como o professor parte do
pressuposto que todos os alunos dedicam-se plenamente a ele, ele as
marca em horários inconvenientes (levando a remarcação em muitos casos).
Tanto a marcação, quanto a remarcação de horários são enviadas via email
a todos os alunos da disciplina.

    Em suma, tudo é enviado por email e nada é postado no sítio da disciplina
(sim! Há um sítio), lotando as caixas de entrada.

\end{document}
