\newacronym{acm}{ACM}{Association for Computing Machinery}
\newglossaryentry{email}{name={E-mail}, description={Correio eletrônico, termo usualmente utilizado para denotar a mensagem eviada por este meio.}}
\newglossaryentry{cabecalho}{name={cabeçalho}, description={Parte do e-mail que contem informações suplementares de transmissão. Entre seus campos, são encontrados endereço do emissor, endereço do receptor, endereço de resposta, data de emissão, tipo do conteúdo e assunto.}}
\newglossaryentry{falsopos}{name={falso-positivo}, description={Classificação errônea na qual, para este contexto, um e-mail legítimo é classificado como spam}}
\newglossaryentry{captcha}{name={CAPTCHA}, description={\emph{Completely Automated Public Turing test to tell Computers and Humans Apart}. Teste criado para diferenciar seres humanos de máquinas. Consiste em imagens de mensagens distorcidas para evitar a interpretação automática por máquinas.}}
\newglossaryentry{spambot}{name={spambot}, description={Bot projetado para enviar spam de forma massiva, automaticamente.}}
\newglossaryentry{bot}{name={bot}, description={Softwares criados para realizar alguma tarefa de forma automatizada.}}
\newglossaryentry{cadmarkov}{name={Cadeia de Markov}, description={Conjunto de estados no qual um processo ocorre. O processo inicia em um estado e se move sucessivamente, transicionando entre estados, a cada passo. O estado para o qual o processo se move depende unicamente, de forma probabilística, do estado em que ele se encontra atualmente.}}
\newglossaryentry{crm114}{name={\emph{CRM114}, description={Software anti-spam baseado em técnicas estatísticas para a filtragem e classificação de dados. Seu código fonte na linguagem C é disponibilizado sob a licença \emph{GPL}}}}
\newglossaryentry{gpl}{name={\emph{GPL}, description={\emph{General Public License}. Licença para software livre.}}}
\newglossaryentry{whitelist}{name={whitelist}, description={Lista de endereços de e-mail de remetentes legítimos, previamente validados.}}
\newglossaryentry{blacklist}{name={blacklist}, description={Lista de mensagens classificadas como spam.}}
\newacronym{mta}{MTA}{Mail Transfer Agent}{description={Software que transfere mensagens de correio eletrônico de um cliente para outro, baseado em uma arquitetura cliente-servidor.}}
\newglossaryentry{threshold}{name={threshold}, description={Valor utilizado como limitante para uma classificação.}}
